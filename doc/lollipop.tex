\section{Lollipop mesh}
We shall model the cerebral aqueduct as a spehere of radius $R$ sitting on top
of a cylinder with radius $r$, $r<R$ and length $l$. A possible approach is to
construct such a domain by rotating a curve. However, if the curve is taken as a
union of a circular arc and a straight segment, see Figure \ref{fig:sharp}, the 
region where the two curves meet has a discontinous gradient. The simple curve
thus leads to unphysical features and a new idea is needed.
%
\begin{figure}[h!]
\centering
  \includegraphics[width=\textwidth]{img/sharp.ps}
\caption{Constructing the domain by rotating pictured curve leads to
discontinuous (tangential) gradient on the surface.
}
\label{fig:sharp}
\end{figure}

To remove the sharp features we propose to custruct the curve by joining the
previously suggested circular arc and the straight segment with a new circular
arc, cf. Figure \ref{fig:smooth}. The additional curve is constructed as
follows:
%
\begin{enumerate}
  \item Let $q$ such that $r<q<R$ is given. We wish to constuct a circular arc
    which starts at point $A=(q, \sqrt{R^2-q^2})\in\Gamma_a$ and ends at the
    straight segment. We define $\mathcal{C}=\set{(x, z); x^2+z^2=R^2}$ and
    $\mathcal{L}=\set{(x, z); x=r}$.
  \item Let $C$ be the intersection point of line $\mathcal{L}$ and the tangent of
    the circle $\mathcal{C}$ at point $C$. Then $d=\semi{C-A}$. Observe from
    Figure \ref{fig:smooth} that $\alpha=\angle BAC=\alpha$ and
    $\semi{A-B}=q-r$. The $d$ can be computed from $\cos\alpha=\semi{A-B}/d$.
    The intersection point itself is then $C=B-(0, h)$, where
    $h/(q-r)=\tan\alpha$.
  \item Let $D=C-(0, d)$ and the point shall be where the new arc intersects
    $\mathcal{L}$. To construct the arc its radius $x$ and center $P$ are needed.
    Note that $P$ must lie on a perpendicular bisector of $AD$.
  \item To make $P$ unique we set it as the intersect of the bisector
    and the line defined by the origin and $A$. The $P=(R+x)(\sin\alpha,-\cos\alpha)$
  \item It remains to find $x$. Here, we observe that $\angle
    ACD=\pi-(\pi/2-\alpha)=\pi/2+\alpha$ and define $2\gamma=\angle ACD$.
    The distance can be computed from the triangle $ACP$ ($CP$ is the hypotenuse)
    as $x/d=\tan{\gamma}$.
\end{enumerate}
\begin{figure}[h!]
\centering
  \includegraphics[width=\textwidth]{img/smooth.ps}
\caption{The curve leading to a smooth domain with continous tangential gradient
  on a whole surface.}
  \label{fig:smooth}
\end{figure}

In the presented algorithm any $q\in(r, R)$ is admissible. An iteresting
question is whether some meaningful objective functional could be defined such 
that it would yield an optimal value of $q$. Of course, it might very well be
that $q$ is a free parameter chosen, e.g. based on the resolution of the mesh; $q$
close to $r$ results in large curvature in the joint region and small mesh size
is thus needed to resolve the junction.

We note that the construction was motivated by a similar problem where a
vertical and horizontal segments are to be joined smoothly. In this case it is
natural to use a circular arc with $\tfrac{\pi}{4}$ sector to join the segments.
Our construction has this problem as a special case. However, there might be
multiple algorithms which collapse to this solution.

